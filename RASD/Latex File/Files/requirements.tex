\subsection{External Interface Requirements}
\subsubsection{User Interfaces}
\subsubsection{Hardware Interfaces}
\subsubsection{Software Interfaces}
\subsubsection{Communication Interfaces}


\subsection{Functional  Requirements}
Evrey function shoud work olnly after succesful login.

\begin{enumerate}
  \goal{1} Allow users to notify authorities about traffic violations %%%%%%%%%%%%%%%%%%%%
  \req{1} User must be able to choose the kind of violation from a list
  %%%thinkkkkkkkk of more%%%%%%%%%%%%



  \goal{2} Allow users to send pictures with metadata of violations %%%%%%%%%%%%%%%%%%%%%%%%%%%%%%%
  \req{1} Application should access the camera
  \req{2} Date, time and position should be automatically added to the violation reported
  \req{}  We shoud require the user to send again a picture in case the plate is not visible
  \req{}  The user must be able to select the veichle to report in case there are other veichles in picture

  \goal{3} Allow users to mine information recorded  %%%%%%%%%%%%%%%%%%%%%%%%%%%%%%%%%%%
  \req {1} Application must be able to count occurrency of violations
  \req {2}  Application must be able to count violation for each veichle
  \req {3} Application should show streets with highest frequency of violation
  \req {}  Application should show the first $n$ (input by user) veichles with the highest number of violations



%%%it's amess from here: FIXXX%%%%%%%%%

  \req{3} The type of violation should be clear in the picture.
  \goal{3} Be sure every information uploaded is never altered
  \goal{4} Automatically add metadata to the reported pictures
  \goal{5} allow users to mine information recorded
  \req{} All individual users who have signed up for mining the information
  \goal{} have at least two different  priviledge for mining data
  \goal{7} generate traffic tickets
  \goal{8} Autorities can see the the licence plates of violators, regular users cannot



\subsubsection{Use Cases}
This section contains all the use cases initially described with the use cases UML model, then the most important Use Case have their own table which provide further details such as:  involved actors, entry conditions,  flow of events, exit conditions and exceptional conditions.


%%%%big pict here %%%%%%%%%%%%



\begin{itemize}
		%%%%%%%%%%%%%%%%%%%%%%%%%%%%%%%%%%
		% DATA4HELP USE CASE SPECIFICATION
		%%%%%%%%%%%%%%%%%%%%%%%%%%%%%%%%%%
		\textbf{ID}: \ucas{1} \\
		\textbf{Name}: Sign-Up \\
		\textbf{Actor}: Guest \\
		\textbf{Entry conditions}:
		\begin{enumerate}
			\item{A citizen who wants to use the service}
		\end{enumerate}
		\textbf{Event flow}:
		\begin{enumerate}
			\item{The guest reaches the registration page containing the relative form}
			\item{The guest fills up the form and clicks on ”Sign up” to complete the process}
			\item{The system redirects the user to his profile page and sends a confirmation email}

		\end{enumerate}
		\textbf{Exit conditions}:
		\begin{itemize}
			\item{The guest has successfully registered in the system.}
		\end{itemize}
		\textbf{Exceptions}:
		\begin{enumerate}
      		\item{The guest left an empty field or typed something wrong an error message is displayed and the user is asked to fill the form again.}
 		   \end{enumerate}
		\rule{\linewidth}{0.4pt}
    %%%%%%%%%%%%%%%%%%%%%%%%%%%%%%%%%%%%%%%%%%%%%%%%%%%%%%%%%%%%%%%%%%%%%%%

		\textbf{ID}: \\
		\textbf{Name}: Login \\
		\textbf{Actor}: User \\
		\textbf{Entry conditions}:
		\begin{enumerate}
			\item{The user has already registered.}
		\end{enumerate}
		\textbf{Event flow}:
		\begin{enumerate}
			\item{The user reaches the login page containing the relative form}
			\item{The user types the username and password in the login form and click on ”Login” button.}
			\item{The system redirects the user to the application homepage.}

		\end{enumerate}
		\textbf{Exit conditions}:
		\begin{itemize}
			\item{The user has access to the application functionalities}
		\end{itemize}
		\textbf{Exceptions}:
		\begin{enumerate}
      		\item{Username and password didn’t correspond or the username didn’t exist, an error message is displayed and the user is asked to fill the login form again.}
 		   \end{enumerate}
		\rule{\linewidth}{0.4pt}
    %%%%%%%%%%%%%%%%%%%%%%%%%%%%%%%%%%%%%%%%%%%%%%%%%%%%%%%%%%%%%%%%%%%%%%%
    \item{  } \\
		\textbf{ID}: \Recover Password  \\
		\textbf{Name}: Login \\
		\textbf{Actor}: User \\
		\textbf{Entry conditions}:
		\begin{enumerate}
			\item{The user has already registered.}
		\end{enumerate}
		\textbf{Event flow}:
		\begin{enumerate}
			\item{The user reaches the login page containing the relative form}
			\item{The user clicks on ”Password recovery” button and is redirected to the password recovery page.}
			\item{The user inserts his email and clicks on ”reset password”.}
			\item{The system sends an email to the user with a link and instruction to reset the password.}
			\item{The user chooses and types a new password and confirms.}
			\item{The application check whether the entered password is strong enough or not.}
			\item{The system redirects the user to the login page.}

		\end{enumerate}
		\textbf{Exit conditions}:
		\begin{itemize}
			\item{The user has changed his password}
		\end{itemize}
		\textbf{Exceptions}:
		\begin{enumerate}
      		\item{The inserted email doesn't match any user in the database, it is displayed an error message and the user is asked to retype a valid email.}
 		   \end{enumerate}
		\rule{\linewidth}{0.4pt}
    %%%%%%%%%%%%%%%%%%%%%%%%%%%%%%%%%%%%%%%%%%%%%%%%%%%%%%%%%%%%%%%%%%%%%%%
		\textbf{ID}: \usecase{3} \\
		\textbf{Name}: Mine information - streets \\
		\textbf{Actor}: User  \\
		\textbf{Entry conditions}:
		\begin{enumerate}
			\item{User is logged in}
		\end{enumerate}
		\textbf{Event flow}:
		\begin{enumerate}
			\item{User enters the section "Explore data"}
			\item{The system asks which kind of data the user wants to know}
      \item{The user chooses to get the data about streets with highest frequency of violations}
      \item{The system queries in descending order the table where for each streets is associated the count of violations }
      \item{The system will report in a tabular way the name of streets and the count of occurred violations}
      \item{If the user scrolls down the system will offer the chance to load more rows}
		\end{enumerate}

		\textbf{Exit conditions}:
          \item{User wants to go back to "Explore data" area}
		\textbf{Exceptions}:
    \begin{enumerate}
      \item{If there are no records the app will report no data available message}
    \end{enumerate}
		\rule{\linewidth}{0.4pt}
    %%%%%%%%%%%%%%%%%%%%%%%%%%%%%%%%%%%%%%%%%%%%%%%%%%%%%%%%%%%%%%%%%%%%%%%
		\textbf{ID}: \usecase{3a} \\
		\textbf{Name}: Mine information by Authority - offenders \\
		\textbf{Actor}: AuthorityUser   \\
		\textbf{Entry conditions}:
		\begin{enumerate}
			\item{AuthorityUser is logged in}
		\end{enumerate}
		\textbf{Event flow}:
		\begin{enumerate}
			\item{AuthorityUser enters the section "Explore data"}
			\item{The system asks which kind of data the AuthorityUser wants to know}
      \item{The AuthorityUser chooses to get the data about veichles that committed the highest number of violations}
      \item{The system queries the table where for each licence plate is associated the count of violations }
      \item{The system will report in a tabular way the plate of the veichle and the count of violations committed}
      \item{If the AuthorityUser scrolls down, the system will offer the chance to load more rows}
		\end{enumerate}
		\textbf{Exit conditions}:
    \begin{enumerate}
      \item{AuthorityUser wants to go back to "Explore data" area}
    \end{enumerate}
		\textbf{Exceptions}:
		\begin{enumerate}
			\item{If there are no records the app will report no data available message}
		\end{enumerate}
		\rule{\linewidth}{0.4pt}

    %%%%%%%%%%%%%%%%%%%%%%%%%%%%%%%%%%%%%%%%%%%%%%%%%%%%%%%%%%%%%%%%%%%%%%%
    %%%%%%%%%%%%%%%%%%%%%%%%%%%%%%%%%%%%%%%%%%%%%%%%%%%%%%%%%%%%%%%%%%%%%%%
    \textbf{ID}: \usecase{3a} \\
    \textbf{Name}: Mine information by EndUser - offenders \\
    \textbf{Actor}: EndUsers   \\
    \textbf{Entry conditions}:
    \begin{enumerate}
      \item{User is logged in}
    \end{enumerate}
    \textbf{Event flow}:
    \begin{enumerate}
      \item{User enters the section "Explore data"}
      \item{The system asks which kind of data the user wants to know}
      \item{The User chooses to get the data about veichles that committed the highest number of violations}
      \item{The system queries the table where for each licence plate is associated the count of violations }
      \item{The system will report in a tabular way an anomymized identifier of the veichle and the count of violations committed if the request comes from a Regular User}
      \item{The system will report in a tabular way the plate of the veichle and the count of violations committed if the request comes from an Authority User}
      \item{If the User scrolls down the system will offer the chance to load more rows}
    \end{enumerate}
    \textbf{Exit conditions}:
    \begin{enumerate}
      \item{User wants to go back to "Explore data" area}
    \end{enumerate}
    \textbf{Exceptions}:
    \begin{enumerate}
      \item{If there are no records the app will report no data available message}
    \end{enumerate}
    \rule{\linewidth}{0.4pt}

    %%%%%%%%%%%%%%%%%%%%%%%%%%%%%%%%%%%%%%%%%%%%%%%%%%%%%%%%%%%%%%%%%%%%%%%

		\textbf{ID}: \usecase{4} \\
		\textbf{Name}: Ticket approval  \\
		\textbf{Actor}: AuthorityUser   \\
		\textbf{Entry conditions}:
		\begin{enumerate}
			\item{A new violation is inserted in database}
      \item{AuthorityUser logged in}
		\end{enumerate}
		\textbf{Event flow}:
		\begin{enumerate}
      \item{Every time a new violation is created by a EndUser the system will create automatically a ticket to be approved}
			\item{AuthorityUser enters the section "Tickets"}
			\item{AuthorityUser enters the section "Approve Tickets"}
      \item{The System will show the list of tickets available for approval }
      \item{AuthorityUser checks the data about ticket}
		\end{enumerate}
		\textbf{Exit conditions}:
    \begin{enumerate}
      \item{User wants to go back to "Ticket" area}
      \item{AuthorityUser approves the ticket}
      \item{AuthorityUser doesn't approve the ticket}
    \end{enumerate}
		\textbf{Exceptions}:
		\begin{enumerate}
			\item{If there are no tickets pending, the app will report no data available message}
		\end{enumerate}
		\rule{\linewidth}{0.4pt}

    %%%%%%%%%%%%%%%%%%%%%%%%%%%%%%%%%%%%%%%%%%%%%%%%%%%%%%%%%%%%%%%%%%%%%%%
    \textbf{ID}: \usecase{5} \\
    \textbf{Name}: Ticket statistics  \\
    \textbf{Actor}: AuthorityUser   \\
    \textbf{Entry conditions}:
    \begin{enumerate}
      \item{}
      \item{AuthorityUser logged in}
    \end{enumerate}
    \textbf{Event flow}:
    \begin{enumerate}

    \end{enumerate}
    \textbf{Exit conditions}:
    \begin{enumerate}

    \end{enumerate}
    \textbf{Exceptions}:
    \begin{enumerate}
      \item{If there are no tickets pending, the app will report no data available message}
    \end{enumerate}
    \rule{\linewidth}{0.4pt}



\subsubsection{User}
\subsubsection{Third party}

\subsubsection{Requirements}
Requirements in order to satisfy the goals
\begin{enumerate}
  \requirement{1} test
\end{enumerate}



%remember to chech if the licence plate exists or not!!!!
\subsection{Performance Requirements}

\subsection{Design Constraints}%%%%%%%%%%%%%%%%%%%T.

\subsubsection{Standards compliance}
The app shoudld be available for the two main operating systems of smartphones: Android Os and Apple iOS.\\
The traffic violations which can be reported should be compliant to the local traffic code where the app will be used.\\
For an use in Italy the app should be compliant to the "Codice della Strada", in particular parking violations are reported in Art. 157.\\


\subsubsection{Hardware limitations}
The app will have a server side and a client side (smartphone).
On server side limitations can be the size of available storage and the bandwidth.
On smartphone side we have the network connectivity (3G/4G connection) and GPS limitations in some areas.

\subsubsection{Any other constraint}
Application should be compliant to European GDPR and don't track users.

\subsection{Software System Attributes} %%%%%%%%%%%%%@S.
\subsubsection{Simple User Interface}
The user interface has to be as simple and intuitive as possible, the application should allow an average user to set up an account and start using the application understanding its functionality in no more than a dozen minutes. In addition there should be a complete tutorial to makes it easy using the application.

\subsubsection{Reliability}
The application provides a reliable service in which individual users can easily log in and report the violations in the most optimal way. Furthermore it Warranties that the chain of custody of the information coming from the users is never broken, and the information is never altered. This would provide a secure and reliable system. In addition, if the license plate	is not readable from the picture the application should warn the user to send an other photo.
\subsubsection{Availability}
The application must offer the maximum availability, granting its service every day at any time (24/7). The lack of service must be minimal.Reporting violation and taking the information about the vioalation coming from SafeStreets must be active every day at any time. The lack of service is acceptable only if it is due to maintenance. In this case, users must receive a warning 48 hours before.
\subsubsection{Security}
The application need to be safe and it does not have particular security concerns except the ones related to unauthorized login. The login of Users and especially of authorities must be very safe to avoid reporting. Moreover, the means of communication must be encrypted to save the confidentiality of information sent to SafeStreets.
\subsubsection{Maintainability}
The application will be maintained and designed in such a way it makes it easier to maintain and it shoul be understandable for both the users and the authorities. Furthermore, the system will put effort in keeping the live data services (such as highlighting the streets with the highest frequency of violations or the vehicles that commit the most violation) always online.
\subsubsection{Portability}
Portability of user data from a device to another is possible by entering personal login data. Also the application will be able to run for devices with different operating systems. Trackme wants to focus on the both Android iOS market and Apple iOS , because Android is the largest OS in the world and it is expected that the market share of Apple iOS will increase in the coming years.
