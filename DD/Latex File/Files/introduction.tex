\subsection{Purpose}
This is the Design Document (DD) of SafeStreet application. The purpose of this document is to discuss more technical aspects regarding architectural and design choices that must be made, so as to follow well-oriented implementation and testing processes.

\subsection{Scope}
The scope of the SafeStreet application can be found in the RASD.

\subsection{Definitions,  acronyms,  abbreviations}

\subsubsection{Definitions}
\begin{itemize}
  \item Heatmap : A heatmap is a graphical representation of data that uses a system of color-coding to represent different values
  \item Enduser : a regular citizen which will use the app
  \item Authority user : someone who's working for an authority like police, municipality etc.
  \item Geocoding : the process of converting addresses (like a street address) into geographic coordinates (latitude and longitude)
  \item Reverse geocoding:  the process of converting geographic coordinates into a human-readable address

\end{itemize}

\subsubsection{Acronyms}
\begin{itemize}
  \item ALPR : Automated Licence Plate Recognition
  \item GUI : Graphical User Interface
  \item UI : User Interface
  \item GDPR : EU General Data Protection Regulation
  \item API : Application Programming Interface
  \item REST : REpresentational State Transfer
  \item JSON : JavaScript Object Notation
  \item RASD : Requirements Analysis and Specifications Document
\end{itemize}

\subsubsection{Abbreviations}
\begin{itemize}
  \item APP : The SafeStreet mobile application to be developed
\end{itemize}

\subsection{Revision history}
This is the first released version 9/12/2019.

\subsection{Reference Documents}
\begin{thebibliography}{00}


\bibitem{clean} Robert C. Martin,
Clean Architecture: A Craftsman’s Guide to Software Structure and Design,
Prentice Hall, Year: 2017
ISBN: 0134494164,9780134494166

\bibitem{openalprdoc}OpenALPR Technology Inc. ,
OpenALPR documentation \url{http://doc.openalpr.com}

\bibitem{mongodb} MongoDB Inc,
The MongoDB 4.2 Manual
\url{https://docs.mongodb.com/manual/}

\bibitem{nodejs} Node.js Foundation,
Node.js v13.1.0 Documentation
\url{https://nodejs.org/api/}

\bibitem{express} StrongLoop, IBM, and other expressjs.com contributors,
Express.js website
\url{http://expressjs.com}

\bibitem{flutter} GOOGLE inc,
Flutter Documentation
\url{https://flutter.dev/docs}

\bibitem{GMapGeocode} GOOGLE inc,
 Google Maps Platform Documentation | Geocoding
 \url{https://developers.google.com/maps/documentation/geocoding/start}

\bibitem{GMapsHeat} GOOGLE inc,
 Google Maps Platform Documentation | Heatmap
 \url{https://developers.google.com/maps/documentation/javascript/heatmaplayer}

\end{thebibliography}

\subsection{Document Structure}
This document is divided in five parts.

\begin{enumerate}
  \item \textbf{Introduction}

  \item \textbf{Architectural Deisgn}

  \item \textbf{User Interface Design}

  \item \textbf{Requirements Traceability}

  \item \textbf{Implementation, Integration and Test Plan}

  \item \textbf{Effort spent} contains the tables where we reported for each group member the hour spent working on the project
\end{enumerate}
