\subsection{Purpose}
This is the Requirement Analysis and Specification Document (RASD) of SafeStreet application. Goals of this document are to completely describe the system in terms of functional and non- functional requirements, analyze the real needs of the customer in order to model the system, show the constraints and the limit of the software and indicate the typical use cases that will occur after the release. This document is addressed to the developers who have to implement the requirements and could be used as a contractual basis.

\subsubsection{Description of the given problem}
SafeStreets is a crowd-sourced application that intends to provide users with the possibility to notify authorities when traffic violations occur, and in particular parking violations. The application allows users to send to authorities  pictures of violations, including their date, time and position. Examples of violations are: vehicles parked in the middle of bike lanes, in places reserved for people with disabilities, on foothpats, double parking etc.

SafeStreets stores the information provided by users, completing it with suitable meta-data every time it recieves a picture.
In paricular it is able to read automatically the license plate of a vehicle and store it without asking the user to type it.
Also it stores the type of the violation which is input by the user from a provided list.
Lastly it stores the name of the street where the violation occurred, rereiving it automatically from the geographical position where the user took the picture.
Then the application allows both end users and authorities to mine the information crowd-souced.
Two visualizations are offfered: the first is an interactive map where are highlighted with a gradient color the streets with the highest frequency of violations.
The second is a list of the vehicles that committed the most violations (available only to authority users).

In addition the app offers a service that creates automatically traffic tickets which can be approved and sent to citizens by the local police.  This is done using the data crowd-soucred by the users.
The application guarantees that every picture used to generate a ticket has't been altered.
In addition, the information about issued tickets is used to build statistics.
Two kind of statistics are offered: a list of people who received the highest number of tickets and some trends of the issued tickets over time and the ratio of approved tickets over the violations reported.




\subsection{Scope}




\subsection{Definitions,  acronyms,  abbreviations}

\subsubsection{Definitions}
\begin{itemize}
  \item{Heatmap} : A heatmap is a graphical representation of data that uses a system of color-coding to represent different values
  \item Enduser : a regular citizen which will use the app
  \item Authority user : someone who's working for an authority like police, municipality etc.
  \item Geocoding : the process of converting addresses (like a street address) into geographic coordinates (latitude and longitude)
  \item Reverse geocoding:  the process of converting geographic coordinates into a human-readable address

\end{itemize}

\subsubsection{Acronyms}
\begin{itemize}
  \item ALPR : Automated Licence Plate Recognition
  \item GUI : Graphical User Interface
  \item GDPR : EU General Data Protection Regulation
  \item API : Application Programming Interface
\end{itemize}

\subsubsection{Abbreviations}


\subsection{Revision history}
This is the first released version 10/11/2019.

\subsection{Reference Documents}
\begin{thebibliography}{00}

\bibitem{clean} Robert C. Martin,
Clean Architecture: A Craftsman’s Guide to Software Structure and Design,
Prentice Hall, Year: 2017
ISBN: 0134494164,9780134494166

\bibitem{openalprdoc}OpenALPR Technology Inc. ,
OpenALPR documentation \url{http://doc.openalpr.com}

\bibitem{mongodb} MongoDB Inc,
The MongoDB 4.2 Manual
\url{https://docs.mongodb.com/manual/}

\bibitem{nodejs} Node.js Foundation,
Node.js v13.1.0 Documentation
\url{https://nodejs.org/api/}

\bibitem{express} StrongLoop, IBM, and other expressjs.com contributors,
Express.js website
\url{http://expressjs.com}


\bibitem{GMapGeocode} GOOGLE inc,
 Google Maps Platform Documentation | Geocoding
 \url{https://developers.google.com/maps/documentation/geocoding/start}

\bibitem{GMapsHeat} GOOGLE inc,
 Google Maps Platform Documentation | Heatmap
 \url{https://developers.google.com/maps/documentation/javascript/heatmaplayer}

\end{thebibliography}

\subsection{Document Structure}
This document is divided in five parts.

\begin{enumerate}
  \item \textbf{Intrduction}

  \item \textbf{Architectural Deisgn}

  \item \textbf{User Interface Design}

  \item \textbf{Requirements Traceability}

  \item \textbf{Implementation, Integration and Test Plan}

  \item \textbf{Effort spent} contains the tables where we reported for each group member the hour spent working on the project
\end{enumerate}
