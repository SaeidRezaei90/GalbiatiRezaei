\subsection{Purpose}


\subsection{Scope}

\subsubsection{Description of the given problem}


\subsubsection{Goals}
\begin{enumerate}
  \goal{1} Allow users to notify authorities about traffic violations
  \goal{2} Allow users to send pictures with metadata of violations
  \goal{3} Allow users to mine information recorded
  \goal{4} Have at least two different  priviledge for mining data
  \goal{5} Generate traffic tickets
  \goal{6} Generate statistics about issued tickets
  \goal{7} Be sure every information uploaded is never altered
\end{enumerate}



\subsection{Definitions,  acronyms,  abbreviations}

\subsubsection{Definitions}
\begin{itemize}
  \item Heatmap : A heatmap is a graphical representation of data that uses a system of color-coding to represent different values

\end{itemize}




\subsubsection{acronyms}
\begin{itemize}
  \item ALPR : Automated Licence Plate Recognition
  \item GUI : Graphical User Interface
\end{itemize}

\subsubsection{abbreviations}


\subsection{Revision history}


\subsection{Reference Documents}
\begin{thebibliography}{00}
\bibitem{openalprdoc}OpenALPR Technology Inc. ,
OpenALPR documentation \url{http://doc.openalpr.com}

\bibitem{codicestrada} Ministero delle infrastrutture e dei Trasporti,
DECRETO LEGISLATIVO 30 aprile 1992, n. 285 Nuovo codice della strada,
 \url{https://www.normattiva.it/uri-res/N2Ls?urn:nir:stato:decreto.legislativo:1992-04-30;285!vig=}

\bibitem{GMapGeocode} GOOGLE inc,
 Google Maps Platform Documentation | Geocoding
 \url{https://developers.google.com/maps/documentation/geocoding/start}

\bibitem{GMapsHeat} GOOGLE inc,
 Google Maps Platform Documentation | Heatmap
 \url{https://developers.google.com/maps/documentation/javascript/heatmaplayer}

\end{thebibliography}


\subsection{Document Structure}
